\documentclass[12pt]{article}
\usepackage{geometry}
\geometry{verbose, letterpaper, tmargin = 2.54cm, bmargin = 2.54cm,
  lmargin = 2.54cm, rmargin = 2.54cm}
\geometry{letterpaper}
\usepackage{amsmath}
\usepackage{fixltx2e}

\begin{document}

\section{Log-difference model}

The log-difference model has several pros and cons over the undifferenced
model. Negatives are that it uses a smaller subset of the data; positives are
that it doesn't require estimating people-specific random effects (n = 3000+)
and removes possible time effects. The basic regression model is 

\begin{equation}
  { Y }_{ i,t } \sim \mathrm{N}(\hat { { Y }}_{ i,t } ,\sigma^2 )
\end{equation}

\subsection{Model for the mean}

The model's expectation is written as a hierarchical random effects linear
model with random effects in the intercept and slope. 

\begin{equation}
{ \hat { Y }  }_{ t,i }={ B }_{ 0,j }+{ B }_{ 1,j }\cdot { s }_{ t,i }+{o}_{ t,i}
\end{equation}

\noindent
where ${ B }_{ 0,j }$ is a normally distributed random effect intercept in the
mean (j subscripting different strategies). The interpretation of ${B}_{ 0,j}$
is that each after log-differencing, each strategy has different effects on
changes in revenue (some tend to be winners, others tend to be losers). Initial
versions of this model included estimating ${ B }_{ 0,j }$, which were largely
centered on 0 (as expected), so these random effects were set to 0.  \newline  
  
The covariate ${ s }_{ t,i }$ is input as data, and represents the percent
change in species diversity over the same time period that the percent change
in revenue is being modeled, $\log\left( \frac { { div }_{ t+1,i } }{ { div }_{
t,i } }  \right)$ where ${ div }_{ t,i }$ is the effective diversity of
individual $i$ at time $t$. The coefficient ${ B }_{ 1,j }$ is a random effect
on the slope, by strategy, representing how much an X\% change in species
diversity within a strategy translates to a Y\% change in revenue. 

\subsection{Offset}

To account for variable effort over time (changes in revenue could be
attributed to diversification, but also to more fishing) we included an offset
term ${o}_{t,i}$, representing the percent change in landing dates between
years. We experimented with different versions of the offset and when included
in the model as a predictor, the coefficient is estimated to be ~ 1, so this is
a reasonable assumption. The offset is calculated as $\log\left( \frac { { day
}_{ t+1,i } }{ { day }_{ t,i } }  \right)$. \newline

\subsection{Equation for the variance}

Finally, the scale of the residual error term is also modeled as a linear
function of covariates, 

\begin{equation}
{ \sigma  }_{ t,i }=
\sqrt { \exp \left( \gamma _{ 0,j }+
{ \gamma  }_{ 1,j }\cdot { s }_{ t,i } \right)  }
\end{equation}

\noindent
where the $\gamma_{ 0,j }$ and $\gamma_{ 1,j }$ represent random intercepts and
slopes in the strategies. I had been interpreting the intercepts as the
between-group variability in strategies (some strategies will have more
variable \% change in revenue than others, similar to some stocks having higher
variability in returns). I was interpreting the random slope $\gamma_{ 1,j }$
as the effect of within-strategy diversification; in other words, the effect
of increases in species diversification varied among different strategies. 

\end{document}
