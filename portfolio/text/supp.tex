\documentclass[12pt]{article}
\usepackage{geometry}
\geometry{verbose, letterpaper, tmargin = 2.54cm, bmargin = 2.54cm,
  lmargin = 2.54cm, rmargin = 2.54cm}
\geometry{letterpaper}
\usepackage{graphicx}
\usepackage{amssymb}
\usepackage{amsmath}
\usepackage{amsfonts}
\usepackage{setspace}
\usepackage{booktabs}
\usepackage{verbatim}
\usepackage{amsmath}
\usepackage{url}
\usepackage{mathtools}
\usepackage{bm}
\usepackage{lineno}
\usepackage{xcolor}
\renewcommand\linenumberfont{\normalfont\tiny\sffamily\color{gray}}
\modulolinenumbers[2]

% Linux Libertine:
\usepackage{textcomp}
\usepackage[sb]{libertine}
\usepackage[varqu,varl]{inconsolata}% sans serif typewriter
\usepackage[libertine,bigdelims,vvarbb]{newtxmath} % bb from STIX
\usepackage[cal=boondoxo]{mathalfa} % mathcal
%\useosf % osf for text, not math
\usepackage[supstfm=libertinesups,%
  supscaled=1.2,%
  raised=-.13em]{superiors}

\mathchardef\mhyphen="2D % math hyphen

\textheight 22.0cm

\usepackage[round,sectionbib]{natbib}
\bibpunct{(}{)}{;}{a}{}{;}

\setlength\parskip{0.10in}
\setlength\parindent{0in}

% Fix line numbering and align environment
% http://phaseportrait.blogspot.ca/2007/08/lineno-and-amsmath-compatibility.html
\newcommand*\patchAmsMathEnvironmentForLineno[1]{%
  \expandafter\let\csname old#1\expandafter\endcsname\csname #1\endcsname
  \expandafter\let\csname oldend#1\expandafter\endcsname\csname end#1\endcsname
  \renewenvironment{#1}%
     {\linenomath\csname old#1\endcsname}%
     {\csname oldend#1\endcsname\endlinenomath}}%
\newcommand*\patchBothAmsMathEnvironmentsForLineno[1]{%
  \patchAmsMathEnvironmentForLineno{#1}%
  \patchAmsMathEnvironmentForLineno{#1*}}%
\AtBeginDocument{%
\patchBothAmsMathEnvironmentsForLineno{equation}%
\patchBothAmsMathEnvironmentsForLineno{align}%
\patchBothAmsMathEnvironmentsForLineno{flalign}%
\patchBothAmsMathEnvironmentsForLineno{alignat}%
\patchBothAmsMathEnvironmentsForLineno{gather}%
\patchBothAmsMathEnvironmentsForLineno{multline}%
}

% remove numbers in front of sections:
\makeatletter
\renewcommand\@seccntformat[1]{}
\makeatother

\title{Supporting Information}

\author{
}
\date{}

\begin{document}

\maketitle

\renewcommand{\thefigure}{S\arabic{figure}}
\renewcommand{\thetable}{S\arabic{table}}

Hierarchical model

\citep{kasperski2013}

\begin{align*}
\mathrm{Pr}(\nu_i < 10) &\sim \mathrm{Beta}(A_i, B_i)\\
\mu_i &= \mathrm{logit}^{-1}(\alpha
  + \alpha^\mathrm{class}_{j[i]}
  + \alpha^\mathrm{order}_{k[i]}
  + \alpha^\mathrm{species}_{l[i]}
  + \bm{X_i} \bm{\beta}),
  \: \\
  &\quad \, \text{for } i = 1, \dots, 609\\
A_i &= \phi_\mathrm{disp} \mu_i\\
B_i &= \phi_\mathrm{disp} (1 - \mu_i)\\
\alpha^\mathrm{class}_j &\sim
  \mathrm{Normal}(0, \sigma^2_{\alpha \; \mathrm{class}}),
  \: \text{for } j = 1, \dots, 6\\
\alpha^\mathrm{order}_k &\sim
  \mathrm{Normal}(0, \sigma^2_{\alpha \; \mathrm{order}}),
  \: \text{for } k = 1, \dots, 38\\
\alpha^\mathrm{species}_l &\sim
  \mathrm{Normal}(0, \sigma^2_{\alpha \; \mathrm{species}}),
  \: \text{for } l = 1, \dots, 301,
\end{align*}

Negatives are that it uses a smaller subset of the data; 

positives are
that it doesn't require estimating people-specific random effects (n = 3000+)
and removes possible time effects. The basic regression model is 

\begin{equation}
  { Y }_{ i,t } \sim \mathrm{N}(\hat { { Y }}_{ i,t } ,\sigma^2 )
\end{equation}

\subsection{Model for the mean}

The model's expectation is written as a hierarchical random effects linear
model with random effects in the intercept and slope. 

\begin{equation}
{ \hat { Y }  }_{ t,i }={ B }_{ 0,j }+{ B }_{ 1,j }\cdot { s }_{ t,i }+{o}_{ t,i}
\end{equation}

\noindent
where ${ B }_{ 0,j }$ is a normally distributed random effect intercept in the
mean (j subscripting different strategies). The interpretation of ${B}_{ 0,j}$
is that each after log-differencing, each strategy has different effects on
changes in revenue (some tend to be winners, others tend to be losers). Initial
versions of this model included estimating ${ B }_{ 0,j }$, which were largely
centered on 0 (as expected), so these random effects were set to 0.  \newline  
  
The covariate ${ s }_{ t,i }$ is input as data, and represents the percent
change in species diversity over the same time period that the percent change
in revenue is being modeled, $\log\left( \frac { { div }_{ t+1,i } }{ { div }_{
t,i } }  \right)$ where ${ div }_{ t,i }$ is the effective diversity of
individual $i$ at time $t$. The coefficient ${ B }_{ 1,j }$ is a random effect
on the slope, by strategy, representing how much an X\% change in species
diversity within a strategy translates to a Y\% change in revenue. 

\subsection{Offset}

To account for variable effort over time (changes in revenue could be
attributed to diversification, but also to more fishing) we included an offset
term ${o}_{t,i}$, representing the percent change in landing dates between
years. We experimented with different versions of the offset and when included
in the model as a predictor, the coefficient is estimated to be $\sim$ 1, so this is
a reasonable assumption. The offset is calculated as $\log\left( \frac { { day
}_{ t+1,i } }{ { day }_{ t,i } }  \right)$. \newline

\subsection{Equation for the variance}

Finally, the scale of the residual error term is also modeled as a linear
function of covariates, 

\begin{equation}
{ \sigma  }_{ t,i }=
\sqrt { \exp \left( \gamma _{ 0,j }+
{ \gamma  }_{ 1,j }\cdot { s }_{ t,i } \right)  }
\end{equation}

\noindent
where the $\gamma_{ 0,j }$ and $\gamma_{ 1,j }$ represent random intercepts and
slopes in the strategies. I had been interpreting the intercepts as the
between-group variability in strategies (some strategies will have more
variable \% change in revenue than others, similar to some stocks having higher
variability in returns). I was interpreting the random slope $\gamma_{ 1,j }$
as the effect of within-strategy diversification; in other words, the effect
of increases in species diversification varied among different strategies. 

\bibliographystyle{apalike}
\bibliography{refs}

\clearpage

\begin{figure}[htbp]
\begin{center}
\includegraphics[width=\textwidth]{../figs/stan-str-posteriors-dot.pdf}
\caption{Caption here}
\label{fig:ts}
\end{center}
\end{figure}

\end{document}
