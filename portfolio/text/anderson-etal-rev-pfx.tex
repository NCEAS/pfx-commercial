\documentclass[9pt,twocolumn,twoside,lineno]{pnas-new}
% Use the lineno option to display guide line numbers if required.
% Note that the use of elements such as single-column equations
% may affect the guide line number alignment.

\templatetype{pnasresearcharticle} % Choose template
% {pnasresearcharticle} = Template for a two-column research article
% {pnasmathematics} = Template for a one-column mathematics article
% {pnasinvited} = Template for a PNAS invited submission

\title{Benefits and risks of diversification for individual fishers}
% \title{Diversification trade-offs for individual fishers}

% Use letters for affiliations, numbers to show equal authorship (if applicable) and to indicate the corresponding author
\author[a]{Sean C. Anderson}
\author[b]{Eric J. Ward}
\author[b]{Andrew O. Shelton}
\author[c]{Milo D. Adkison}
\author[c]{Anne H. Beaudreau}
\author[d]{Richard E. Brenner}
\author[e]{Alan C. Haynie}
\author[d]{Jennifer C. Shriver}
\author[f]{Jordan T. Watson}
\author[d]{Benjamin C. Williams}

\affil[a]{School of Aquatic and Fishery Sciences, Box 355020,
University of Washington, Seattle, WA 98195, USA}

\affil[b]{Conservation Biology Division, Northwest Fisheries Science Center,
National Marine Fisheries Service, National Oceanographic and
Atmospheric Administration, 2725 Montlake Blvd E, Seattle, WA 98112, USA}

\affil[c]{College of Fisheries and Ocean Sciences, University of Alaska Fairbanks,
17101 Point Lena Loop Rd, Juneau, AK, 99801, USA}

\affil[d]{Division of Commercial Fisheries,
Alaska Department of Fish and Game, PO BOX 115526, Juneau, AK 99811, USA}

\affil[e]{Alaska Fisheries Science Center,
National Marine Fisheries Service, National Oceanographic and
Atmospheric Administration, Bldg 4,
7600 Sand Point Way NE, Seattle, WA, 98115, USA}

\affil[f]{Alaska Fisheries Science Center, National Marine Fisheries Service, National Oceanographic and
Atmospheric Administration, Auke
Bay Laboratories, 17109 Pt. Lena Loop Rd., Juneau, AK 99801, USA}

% Please give the surname of the lead author for the running footer
\leadauthor{Anderson}

% Please add here a significance statement to explain the relevance of your work % <120 words
\significancestatement{
Individuals that rely on natural resources for their livelihoods, such as fishers, farmers, and forestry workers, face high levels of income variability. For fishers, catching multiple species has been shown to reduce revenue variability at large scales (vessels and communities) but the individual-level consequences of maintaining catch diversity are unknown. Our work demonstrates that individuals in fisheries targeting a diversity of species and individuals that participate in multiple fisheries buffer income variability compared to less diverse individuals. However, large adjustments in diversification strategies from year-to-year are risky and usually increase revenue variability. The most effective option to reduce revenue variability via diversification --- purchasing additional permits --- is also expensive, often limited by regulations, and therefore unavailable to many. }
% We ask how multiple forms of % species diversification affect individual-level % revenue and revenue variability for Alaskan fishers.

% Please include corresponding author, author contribution and author declaration information
\authorcontributions{
S.C.A, E.J.W, and A.O.S designed the study with contributions from all authors;
J.C.S. and J.T.W. also provided and cleaned the data;
S.C.A. and E.J.W. analyzed the data and wrote the paper with input from all authors.}
\authordeclaration{The authors declare no conflict of interest.}
\correspondingauthor{\textsuperscript{2}To whom correspondence should be addressed. E-mail: sandrsn@uw.edu}

% Keywords are not mandatory, but authors are strongly encouraged to provide them. If provided, please include two to five keywords, separated by the pipe symbol, e.g:
\keywords{
diversity-stability $|$
Bayesian variance-function regression $|$
income variability $|$
natural resource management $|$
ecological portfolio effects
}

\newcommand{\tabstrategies}{1}
\newcommand{\figspokeall}{1}
\newcommand{\figfe}{2}
\newcommand{\figre}{3}
\newcommand{\figmirror}{4}
\newcommand{\fignoeffort}{5}
\newcommand{\figifq}{6}
\newcommand{\figstryrs}{7}
\newcommand{\figresid}{8}
\newcommand{\figresiddownside}{9}

% 250:
\begin{abstract}

Individuals relying on the extraction of natural resources for their livelihood
face high levels of income variability driven by a mix of environmental,
biological, management, and economic factors. Key to managing these industries
is identifying how regulatory actions and individual behaviour affect income
variability, financial risk, and, by extension, the economic stability and the
sustainable use of natural resources. In commercial fisheries, communities and
vessels fishing a greater diversity of species have less revenue variability
than those fishing fewer species. However, it is unclear if these benefits
extend to the actions of individual fishers and how year-to-year changes in
diversification affect revenue and revenue variability. Here, we evaluate two
axes by which fishers in Alaska can diversify fishing activities. We show that,
despite increasing specialization over the last 30 years, fishing a set of
permits with higher species diversity reduces individual revenue variability and
fishing an additional permit is associated with higher revenue and lower
variability. However, increasing species diversity within the constraints of
existing permits has a fishery-dependent effect on revenue and is usually (87\%
probability) associated with increased revenue uncertainty the following year.
Our results demonstrate that the most effective option for individuals to
decrease revenue variability is to participate in additional or more diverse
fisheries. However, this option is expensive, often limited by regulations such
as catch share programs, and consequently unavailable to many individuals. With
increasing climatic variability it will be particularly important that
individuals relying on natural resources for their livelihood have effective
strategies to reduce financial risk.

% (`r g2_dens`% probability)
\end{abstract}

\dates{This manuscript was compiled on \today}
\doi{\url{www.pnas.org/cgi/doi/10.1073/pnas.XXXXXXXXXX}}

\begin{document}

% Optional adjustment to line up main text (after abstract) of first page with line numbers, when using both lineno and twocolumn options.
% You should only change this length when you've finalised the article contents.
\verticaladjustment{-2pt}

\maketitle
\thispagestyle{firststyle}
\ifthenelse{\boolean{shortarticle}}{\ifthenelse{\boolean{singlecolumn}}{\abscontentformatted}{\abscontent}}{}

% If your first paragraph (i.e. with the \dropcap) contains a list environment (quote, quotation, theorem, definition, enumerate, itemize...), the line after the list may have some extra indentation. If this is the case, add \parshape=0 to the end of the list environment.

\input{ms}

\matmethods{

We obtained commercial fisheries landings and revenue data for all target and
non-target species for permit
holders in Alaska from 1985 to 2014 from the Alaskan Commercial Fisheries Entry
Commission (CFEC). These data represent a total of nearly 43 billion 2009 USD.
We adjusted for inflation by setting all revenues to 2009 USD \cite{bea2016}.
We implemented a number of filtering steps to focus on individuals actively
engaged in commercial fishing (Supporting Information). In particular, we
removed permit holders whose median revenue was less than \$10,000 USD to focus
on individuals with substantial income from commercial fishing.

We used combinations of state and federal fishing permits held by an individual
in a specific year to define fishing ``strategies'', representing the species
caught (Supporting Information). For example, we combined permits that were
otherwise the same except for vessel size (e.g.~a longliner fishing sablefish
on a vessel under 60 ft was considered to have the same strategy as a longliner
fishing sablefish on a vessel over 60 ft), and we combined a number of permits
that were caught a single species and differed only in region fished. For
example, someone fishing herring roe in Kodiak and someone fishing herring roe
in Alaska Peninsula were both considered to have a ``herring roe'' permit
strategy. For salmon permits, we formed salmon strategies based on similarities in
gear type and species composition (Supporting Information).
Gear types and permits are closely linked since in most cases a fisher would require a different permit to use different gear. The ability to diversify without changing or adding gear or permits is therefore driven by biological availability and choices by fishers, such as fishing early versus late in a year or fishing in particular locations.
% Response to reviewer comment A5

After this process of combining permits, we were left with 23 unique fishing
permit groups (Table~S\tabstrategies), which we aggregated within each
person-year combination to form permit strategies. For example, if someone
fished halibut and sablefish in the same year, their strategy for that year
would be ``halibut-sablefish''. Specific details on our strategy definitions
are available in the Supporting Information.

Our models focused on species diversity as a possible predictor of revenue and
revenue variability while controlling for effort. For consistency with previous
analyses \cite{vedeld2007, illukpitiya2008, debela2012, kasperski2013,
  sethi2014, cline2017}, we calculated effective species diversity
for permit holder \(i\) and time (year) \(t\) as the inverse of Simpson's
concentration index \(\lambda\) \cite{simpson1949, jost2006} (this index is also referred to in
economics as the Herfindahl-Hirschman Index (HHI) \cite{hirschman1964}) weighted by
revenue \(R\):

\begin{equation}
1/\lambda_{i,t} = 1 / \left(\sum\limits_{s=1}^{n_{i,t}}
  \left(R_{i,s,t}/ \sum\limits_{s=1}^{n_{i,t}} R_{i,s,t}\right)\right)^2,
\end{equation}

\noindent where \(s\) indexes species from \(1\) through \(n\).

We calculated
effort or effective season length as the sum of the season lengths for each
permit each year. Season length for a given permit was calculated as the number
of days between the first instance of fishing a permit and the last instance of
fishing a permit each year. For example, if an individual's season lated for 30 days on a
salmon drift gillnet permit in Southeast Alaska and 40 days on a statewide
longline halibut permit, that individual's season length was calculated as 70
days.
% AH: I imagine at our scale this may not matter much, but did you use actualy fishing days too? If people are fishing another permit at the start and end of their main permit, this would show up as a huge increase in effort, even though it’s really small.

\subsection{Hierarchical model}\label{hierarchical-model}

To jointly model revenue and revenue variability as a function of
individual-level covariates, we extended the basic variance-function regression
to a hierarchical Bayesian variance-function regression model with Gaussian
errors predicting \(\log\) revenue, \(\log(R_{i,t})\), for each fisher \(i\)
and time \(t\):

\begin{equation}
\log(R_{i,t}) \sim \mathrm{N} \left(\mu_{i,t}, \sigma_{i,t}^2 \right).
\end{equation}

\noindent Each year \(t\) of revenue for fisher \(i\) is assigned a strategy
(permit set) indexed by \(j\) (Supporting Information, Table~\tabstrategies)
and \(\mu_{i,t}\) is modelled as:

\begin{equation}
  \begin{split}
\label{eq:mean}
\mu_{i,t} = \beta_{0,j,t} -
  \beta_{1,j} S_{i,t} I_{i,t} +
  \beta_{2,j} S_{i,t} I_{i,t} +
  \beta_3 D_{i,t} -\\
  \beta_4 S_{i,t} D_{i,t} I_{i,t} +
  \beta_5 S_{i,t} D_{i,t} I_{i,t} +
  \log(R_{i,t-1}),
  \end{split}
\end{equation}

\noindent where \(S_{i,t}\) is the log ratio of species diversity from
year-to-year, \(\log \left(\mathrm{div}_{i,t}/\mathrm{div}_{i,t-1}\right)\),
\(D_{i,t}\) is the log ratio of days fished from year-to-year, \(\log
  \left(\mathrm{days}_{i,t}/\mathrm{days}_{i,t-1}\right)\), and \(I_{i,t}\) is
an indicator variable that takes a value \(0\) if \(\mathrm{div}_{i,t} \ge
  \mathrm{div}_{i,t-1}\) (species diversity increasing year-to-year;
generalizing) and takes a value \(1\) if \(\mathrm{div}_{i,t} <
  \mathrm{div}_{i,t-1}\) (species diversity decreasing year-to-year;
specializing). Estimating a separate coefficient for the effect of increasing
diversity and decreasing diversity allows the costs and benefits of these
changes to be asymmetric (Fig.~S\figresid). The negative signs in front of
$\beta_{1,j}$ and $\beta_4$ makes these coefficients interpretable as effects of
increasing specialization (as opposed to decreasing specialization).

We use \(\log(R_{i,t-1}\)) as an offset, and thus our model describes the log
ratio of revenue from year-to-year, i.e., \(\log(R_{i,t}) - \log(R_{i,t-1}) =
  \log(R_{i,t}/R_{i,t-1})\). We model change in revenue because preliminary
analyses revealed non-stationary trends in revenue (resulting from experience
or other unmodelled factors). This approach reduces the
non-stationarity of the time series and allows us to avoid estimating thousands of
mean-revenue intercepts associated with individual fishers.
Because we were modelling year-to-year changes,
we removed person-year combinations without a subsequent year of revenue.

The \(\beta\) terms represent estimated coefficients, some of which are allowed
to vary as ``random effect'' terms. The intercepts \(\beta_{0,j,t}\) are
allowed to vary by strategy-year combinations, are centred on zero, and are
constrained by a normal distribution:

\begin{equation}
\beta_{0,j,t} \sim \mathrm{N} \left(0, \sigma^2_{\beta_{j,t}} \right).
\end{equation}

\noindent This approach, combined with the lack of a global intercept,
constrains the mean change in log revenue across all fishers and years to be
zero. I.e., we assume that the data are stationary after accounting for all
covariates and including an intercept that varies with strategy-year
combinations. Initial versions of the model confirmed that the global intercept
along with strategy-level intercepts would be estimated at almost exactly zero.
The estimated \(\beta_{j,t}\) coefficients represent years when people with a
particular set of permits tended to collectively do better or worse than the
long-term average (Fig.~S\figstryrs). Therefore, an individual's expected
change in revenue (and variability) represents the expectation after accounting
for the general trend in revenue for that set of permits.

The slopes \(\beta_{1,j}\) and \(\beta_{2,j}\) are then allowed to vary by
strategy with estimated means:

\begin{align}
\beta_{1,j} &\sim \mathrm{N} \left(\mu_{\beta_1}, \tau^2_{\beta_1}\right),\\
\beta_{2,j} &\sim \mathrm{N} \left(\mu_{\beta_2}, \tau^2_{\beta_2}\right).
\end{align}

\noindent The \(\beta_{1,j}\) and \(\beta_{2,j}\) coefficients represent how much a 1\%
decrease and 1\% increase, respectively, in the ratio of species diversity from
year-to-year within a strategy translates to a given percent change in the
ratio of revenue from year-to-year. These are after controlling for changes in
effort, \(\beta_3\), and the interaction between changes in effort and changes
in species diversity, \(\beta_4\) and \(\beta_5\).

Whereas traditional hierarchical linear models assume homoscedasticity---that
the residual variance does not vary systematically---our variance-function
regression model allows the residual variance to be heteroscedastic and vary as
a function of predictors. Instead of estimating a single residual error scale
term, \(\sigma\), we model the scale of the residual error with another
hierarchical model with a similar form:

\begin{equation}
\begin{split}
\label{eq:cv}
\sigma_{i,t} =
\exp (\gamma_{0,j} -
\gamma_{1,j} S_{i,t} I_{i,t} +
\gamma_{2,j} S_{i,t} I_{i,t} +
\gamma_3 D_{i,t} -\\
\gamma_4 S_{i,t} D_{i,t} I_{i,t} +
\gamma_5 S_{i,t} D_{i,t} I_{i,t}).
\end{split}
\end{equation}

\noindent Again, \(S_{i,t}\) and \(I_{i,t}\) represent, respectively, the log
ratio of species diversity from year-to-year and an indicator variable for
increasing or decreasing species diversity. We exponentiate the equation to
ensure all scale parameters, \(\sigma_{i,t}\), are positive. Residuals from
the mean component of our model clearly show a ``V''-shaped pattern, which the
variance component of our model can capture (Fig.~S\figresid). Furthermore,
the downside residuals (year-individual data points with lower than expected
revenue) appear qualitatively similar to the complete set of residuals (upside
and downside) (Fig.~S\figresiddownside). Therefore, it seemed appropriate to
model variability as a proxy for revenue risk. Ideally, we would model
a more direct measure of risk such as the probability of bankruptcy from low
revenue; however, data on costs and other sources of income were not available.

The \(\gamma\)'s are estimated coefficients, three of which are allowed to vary
by strategy, constrained by normal distributions:

\begin{align}
\gamma_{0,j}  &\sim \mathrm{N} \left(\eta_0 + \eta_1 M_j + \eta_2 E_j, \tau^2_{\gamma_0}\right),\\
\gamma_{1,j}  &\sim \mathrm{N} \left(\mu_{\gamma_1}, \tau^2_{\gamma_1}\right),\\
\gamma_{2,j}  &\sim \mathrm{N} \left(\mu_{\gamma_2}, \tau^2_{\gamma_2}\right).
\end{align}

\noindent The \(\gamma_{0,j}\) coefficients are modelled with two
strategy-level predictors. The symbols \(M_j\) and \(E_j\) represent the mean
species diversity and mean combined fishing season length (effort) for strategy
\(j\). The coefficients \(\eta_1\) and \(\eta_2\) represent predictors in this
strategy-level regression estimating effects across strategies.

We fit our models with Stan 2.14.1 \cite{standevelopmentteam2016,
  standevelopmentteam2016a} and R 3.3.2 \cite{rcoreteam2016}. Stan implements
the No-U-Turn Hamiltonian Markov chain Monte Carlo algorithm \cite{hoffman2014}
to perform Bayesian statistical inference. We assigned weakly informative
priors on all parameters: \(N(0, 2^2)\) priors on all \(\beta\), \(\gamma\),
and \(\eta\) parameters, and half-\(t(3, 0, 2)\) priors (i.e.~degrees of
freedom of 3 and scale of 2) on all \(\sigma\) and \(\tau\). We ran four chains
and 2,000 iterations and discarded the first 1,000 iterations of each chain as
warm up. We checked for chain convergence visually with trace plots, ensured
that \(\hat{R}<1.05\) (the potential scale reduction factor), and that the
effective sample size, as calculated by the rstan R package
\cite{standevelopmentteam2016}, was greater than 200 for all parameters
\cite{gelman2014}. We verified that our model returned sensible estimates by
comparing our estimates to an ad-hoc two-stage model fit with the lme4 R
package \cite{bates2015}, where we first fit a mixed-effects regression model
to the mean, and then modelled the absolute residuals from that first model in
a second mixed-effects model representing the variance component (Supporting Information).
%We also
%validated the model with simulated data, and were able to recover the true
%values (Supporting Information).

The fisheries data that support the findings of this study, though
confidential, are available from the Commercial Fisheries Entry
Commission (\url{https://www.cfec.state.ak.us/}) and summaries used in
this paper, along with all code, are archived in a publicly available repository at
\url{https://github.com/NCEAS/pfx-commercial/tree/master/portfolio}.

\nocite{bates2015}
} \showmatmethods

\acknow{This work was conducted as part of the National Center for Ecological
  Analysis and Synthesis working group ``Applying portfolio effects to the Gulf
  of Alaska ecosystem: Did multi-scale diversity buffer against the \textit{Exxon
  Valdez} oil spill?''. We thank other participants of the
  working group for helpful discussions. Additional support was provided by a
  David H. Smith Conservation Research Fellowship (S.C.A.) and the NOAA Fisheries'
  Spatial Economics Toolbox for Fisheries (FishSET) Project (J.T.W).}

\showacknow % Display the acknowledgments section

% \pnasbreak splits and balances the columns before the references.
% If you see unexpected formatting errors, try commenting out this line
% as it can run into problems with floats and footnotes on the final page.
\pnasbreak

% Bibliography
\bibliography{refs}

\end{document}
